\section*{\nr.4 \titfour (10 Punkte)}
\begin{enumerate}[(a)]
\item Es gilt
\begin{equation}
	P_{st} = c|{\vec{\pi}_{st}}| = \frac{I}{c} \approx \SI{6,67}{\pascal}
\end{equation}
für die Kraft ergibtt sich daraus
\begin{equation}
	F_{st} = P_{st}A = \SI{3,18E-4}{\newton}
\end{equation}
\item Bei einer Reflexion ist die Impulsänderung doppelt so groß wie bei einer Absorbtion, also $F_{refl} = 2F_{abs} = 2F_{st}$.
\item Für die Dichte des Kügelchens wird $\rho = \SI{1}{\gram \per \centi\metre ^3}$ angenommen. Daraus ergibt sich die Masse bzw. die Gewichtskraft $m = \rho \frac{4}{3} \pi r^3$ und $F_G = mg$. Damit der Laser das Kügelchen gegen die Gewichtskraft halten kann muss gelten
\begin{equation}
	F_G= F_{st} = \frac{IA}{c} \Leftrightarrow I = \frac{cF_G}{A} =\SI{2,943E7}{\watt\per\square\metre}.
\end{equation}
Der Strahlungsdruck ergibt sich als
\begin{equation}
	P_{st} = \frac{I}{c} \approx \SI{0,1}{\newton\per\square\metre}
\end{equation}
Die Strahlungsleistung $\Phi$ ergibt sich als
\begin{equation}
	\Phi = IA \approx \SI{5,2E-3}{\watt}
\end{equation}
\end{enumerate}
