\section*{\nr.1 \titone (10 Punkte)}
\begin{enumerate}[(a)]
\item
Es gilt 
\begin{equation}
  \lambda = \frac{c}{f} = 3\mathrm{m}.
\end{equation}
\item 
Skizze fehlt, hier jedoch eine Beschreibung, wie die Welle aussehen würde:
Das elektrische und das magnetische Feld sind phasengleich, d.h. sie haben am selben Ort die Maxima bzw. ihre Minima. Außerdem stehen sie sowohl senkrecht auf der Ausbreitungsrichtung als auch aufeinander. Dabei ist in Richtung der Ausbreitung das magnetische Feld um $90^\circ$ im Uhrzeigersinn gegenüber dem elektrischen Feld gedreht.

\item Es gilt
\begin{equation}
  \phi=\int\vec B(t) \vec{\mathrm{d}A}
\end{equation}
mit der Definition des mittleren Magnetischen Feldes innerhalb der Antenne und der Symmetrie des Systems ergibt sich
\begin{equation}
  \phi = \int \bar B \cos \omega t \mathrm{d}A=\bar B \cos \omega t \int \mathrm{d}A
\end{equation}
mit $\bar B = \frac{2B_0}{\pi}$.
Daraus folgt
\begin{equation}
  \phi(t)=\frac{B_0\lambda^2}{2\pi}\cos \omega t.
\end{equation}
\item
\begin{equation}
  U_{ind}(t)=-\dot{\phi}=-\frac{B_0\lambda^2}{2\pi}\omega \sin \omega t= -B_0\lambda c \sin \omega t 
\end{equation}
\item 
Mit $E(t)=E_0\sin( \frac{2\pi}{\lambda}x-\omega t)$ ergibt
\begin{equation}
  \int_{\partial A}\vec E(t) \vec{\mathrm{d}s}=-\lambda E_0\sin \omega t = -B_0\lambda c\sin \omega t
\end{equation}
das ist das selbe Ergebnis wie man es auch bereits in Aufgabe d) erhalten hat.


\end{enumerate}