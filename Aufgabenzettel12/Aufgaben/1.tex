\section*{\nr.1 \titone (10 Punkte)}
\begin{enumerate}[(a)]
\item Die Zeit die der Lichtstrahl von A nach B benötigt ergibt sich als
\begin{equation}
  t=\frac{n_1}{c}\sqrt{a^2+x^2}+\frac{n_2}{c}\sqrt{b^2+(d-x)^2}
\end{equation}
mit $\frac{\mathrm{d}t}{\mathrm{d}x}=0$ ergibt sich
\begin{equation}
  \frac{n_1}{c}\frac{x}{\sqrt{a^2+x^2}}=\frac{n_2}{c}\frac{d-x}{\sqrt{b^2+(d-x)^2}}
\end{equation}
woraus direkt
\begin{equation}
  n_1\sin \alpha_1 =n_2\sin \alpha_2
\end{equation}
folgt.

Man betrachte nun den reflektierten Lichtstrahl.
Die Zeit 
\begin{equation}
  t=\frac{x}{c\sin\alpha}+\frac{d-x}{c\sin\beta}
\end{equation}
muss dafür gemäß dem Fermatschen Prinzip extremal werden. Aus $\frac{\mathrm{d}t}{\mathrm{d}x}=0$ folgt nun direkt $\alpha=\beta$, das bekannte Reflektionsgesetz.

\item Dieses Reflektionsgesetz soll nun für die Parametrisierung des Parabolspigels benutzt werden. Seien im folgenden $\alpha$ der Winkel zwischen dem einfallenden Lichtstrahl zum lokalen Lot und $\gamma$ der Winkel zwischen der $y$-Achse und dem reflektierten Strahl.
Es gelten dann mit einfachen geometrischen Überlegungen
\begin{equation}
  \gamma=90^{\circ}-2\alpha,
\end{equation}
\begin{equation}
  \alpha = \arctan\left(\frac{\mathrm{d}x}{\mathrm{d}y}\right)
\end{equation}
sowie 
\begin{equation}
  \tan \gamma=\frac{f-x}{y}.
\end{equation}
Damit ergibt sich 
\begin{align}
  &\arctan\left(\frac{f-x}{y}\right)=\frac{\pi}{2}-2\arctan\left(\frac{\mathrm{d}x}{\mathrm{d}y}\right)\\
  &\iff \frac{f-x}{y}=\cot\left(2\arctan\left(\frac{\mathrm{d}x}{\mathrm{d}y}\right)\right)\\
  &\iff f-x=\frac{y\left(1-\left(\frac{\mathrm{d}x}{\mathrm{d}y}\right)^2\right)}{2\left(\frac{\mathrm{d}x}{\mathrm{d}y}\right)}\\
  &\iff x(y)=f-\frac{y\left(1-\left(\frac{\mathrm{d}x}{\mathrm{d}y}\right)^2\right)}{2\left(\frac{\mathrm{d}x}{\mathrm{d}y}\right)}
\end{align}
Diese Differentialgleichung wird durch die Funktion
\begin{equation}
  x(y)=ay^2+b
\end{equation}
gelöst. Mit $x(0)=0$ und $x(y_f)=f\implies x'(y_f)=\pm1$ ergeben sich $a=\frac{f}{4}$ und $b=0$ und damit die Parametrisierung für den Spiegel
\begin{equation}
  x(y)=\frac{f}{4}y^2
\end{equation}

\end{enumerate}