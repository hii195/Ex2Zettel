\section*{\nr.1 \titone (10 Punkte)}
\begin{enumerate}[(a)]
\item Für die Kapazität gilt
\begin{equation}
  C=\frac{\epsilon_0A}{d}=1.1\cdot 10^{-10}F.
\end{equation}
\item Für den Betrag des elektrischen Feldes gilt
\begin{equation}
  E=\frac{U}{d}=5\cdot 10^5 \frac{V}{m}.
\end{equation}
\item Für die Kraft gilt
\begin{equation}
  F=EQ=E^2\epsilon_0A=0.1N.
\end{equation}
\item Für die Spannung im Kondensator gilt
\begin{equation}
  U=\frac{Qd}{\epsilon_0A},
\end{equation}
also ist für $Q=const.$
\begin{equation}
  U\propto d.
\end{equation}
Daraus folgt, dass $U_1=2U_0=2kV$.
\item Im Kondensator ist allgemein die Energie 
\begin{equation}
  E=\frac{1}{2}CU^2
\end{equation}
gespeichert. 
Die Differenz der jeweiligen Energien ergibt die Arbeit die verrichtet werden muss. Es gilt also
\begin{equation}
  W_Q=\frac{1}{2}U_0^2(C_1-C_0)=\frac{1}{2}U_0^2\epsilon_0A\left(\frac{1}{d_1}-\frac{1}{d_0}\right)=???
\end{equation}
TODO: REST
\end{enumerate}