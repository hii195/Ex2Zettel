\section*{\nr.1 \titone (10 Punkte)}
\begin{enumerate}[(a)]
\item Für die Kapazität gilt
\begin{equation}
  C=\frac{\epsilon_0A}{d}=1.1\cdot 10^{-10}F.
\end{equation}
\item Für den Betrag des elektrischen Feldes gilt
\begin{equation}
  E=\frac{U}{d}=5\cdot 10^5 \frac{V}{m}.
\end{equation}
\item Für die Kraft gilt
\begin{equation}
  F=EQ=E^2\epsilon_0A=0.1N.
\end{equation}
\item Für die Spannung im Kondensator gilt
\begin{equation}
  U=\frac{Qd}{\epsilon_0A},
\end{equation}
also ist für $Q=const.$
\begin{equation}
  U\propto d.
\end{equation}
Daraus folgt, dass $U_1=2U_0=2kV$.
\item Im Kondensator ist allgemein die Energie 
\begin{equation}
  E=\frac{1}{2}CU^2
\end{equation}
gespeichert. 
Die Differenz der jeweiligen Energien ergibt die Arbeit die verrichtet werden muss. Es gilt also
\begin{align}
  W_U&=\frac{1}{2}C_1U_1^2- \frac{1}{2}C_0U_0^2 \\
  &= U_0^2 \left(2C_1-\frac{1}{2}C_0\right)\\
  &= U_0^2 \epsilon_0A\left(\frac{2}{d_1}-\frac{1}{2d_0}\right)\\
  &= U_0^2 \epsilon_0A\left(\frac{1}{d_0}-\frac{1}{2d_0}\right)\\
  &= \frac{U_0^2 \epsilon_0A}{2d_0}\\
  &=1.1\cdot10^{-4}J.
\end{align}
\item Wieder ist die zu leistende Arbeit die Differenz der im Kondensator gespeicherten Energien, womit sich 
\begin{equation}
  W_Q=\frac{1}{2}U_0^2(C_1-C_0)=\frac{1}{2}U_0^2\epsilon_0A\left(\frac{1}{d_1}-\frac{1}{d_0}\right)=5.5 \cdot 10^{-5}J
\end{equation}
ergibt.
\item Trennt man den Kondensator von der Spannungsquelle, so muss die Gesamte Energie, die bei dem Auseinanderziehen benötigt wird mechanisch verrichtet werden, bleibt die Spannungsquelle jedoch angeschlossen, so kann zusätzlich zur mechanischen Energie auch elektrische Energie aus der Spannungsquelle nachfließen, weshalb die benötigte Arbeit geringer ist.
\end{enumerate}