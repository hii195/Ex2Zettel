\section*{\nr.1 \titone (10 Punkte)}
\begin{enumerate}[(a)]
\item
Beim Aufladen fungiert Elektrode 1 als Elektronenquelle, also als Minuspol. Entsprechend ist die Elektrode 2 hier der Pluspol. Beim Entlade laufen die Reaktionen rückwärts ab, d.h. Elektrode 2, also die Bleielektrode fungiert hierbei als Elektronenquelle, welche damit der Minuspol ist. Elektrode 1 nimmt Elektronen auf und ist somit der Pluspol.
\item
Einer einzelne Bleizelle liefert eine Spannung von $U_{z} = \SI {2.1}{\volt}$. Also fällt über 6 in Reihe geschalteten Bleizellen eine Spannung von $U = \SI {12.6}{\volt}$ ab. Die Leistungsfähigkeit entspricht einer Ladung Q. Damit gilt für die gespeicherte Energie
\begin{equation}
	E = Q \cdot U = \SI{50}{\ampere\hour} =\SI {630}{\watt\hour} 
\end{equation}
\item
Aus den Reaktionsgleichungen ist zu sehen, dass in einer Zelle für jedes freiwerdende Elektron ein PbSO$_{4}$ Molekül entsteht. Da ein Elektron im Prinzip jede Zelle durchfließt und dabei in jeder Zelle ein PbSO$_{4}$ Molekül erzeugt muss die Zahl der PbSO$_{4}$ Teilchen sechsmal so hoch sein wie die der Elektronen.
Die Elektronenzahl ergibt sich durch
\begin{equation}
	N_{e} = \frac{Q}{e} = \frac{\SI {50}{\ampere\hour}}{e} \approx \SI {1.17e24}{}.
\end{equation} 
PbSO$_{4}$ hat eine molare Masse von $M(PbSO_{4}) = \SI[per-mode=fraction]{303}{\gram\per\mol}$. Also folgt für die entstehende Masse an PbSO$_{4}$
\begin{equation}
	m_{PbSO_{4}} = 6 \cdot \frac{N_{e} M(PbSO_{4})}{N_{A}} \approx \SI {3.545}{\kilogram}.
\end{equation}
\item
Die Energiedichte ergibt sich für den Bleiakkumulator als
\begin{equation}
	\rho = \frac{E}{m} = \frac{0,8 \cdot \SI {630}{\watt \hour}}{\SI {15}{\kilogram}} = \SI[per-mode=fraction] {33,6}{\watt\hour\per\kilogram}.
\end{equation}
Die Energiedichten von Lithium-Ionen- ($\rho = \SI {120}{\watt\hour\per\kilogram} - \SI {180}{\watt\hour\per\kilogram}$) bzw. Lithium-Polymer-Akkkus (($\rho = \SI {130}{\watt\hour\per\kilogram} - \SI {150}{\watt\hour\per\kilogram}$) liegen deutlich höher. \\(Quelle: http://patona.de/ratgeber/akkuvergleich-die-energiedichte-verschiedener-akkutypens)
\end{enumerate}
