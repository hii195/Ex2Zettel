\section*{\nr.4 \titfour (10 Punkte)}
\begin{enumerate}[(a)]
\item Durch Verwendung von charakteristischen Funktionen lässt sich die Transparenzfunktion $a$ folgendermaßen ausdrücken:
\begin{equation}
a(x,y) = \chi_{(-\infty,\infty)} (y) \left(\chi_{[9d,10d]}(x) + \chi_{[-10d,-9d]}(x) \right)
\end{equation}
Da sich die Funktion in Faktoren von $x$ und Faktoren von $y$ ausdrücken lässt und weil die $y$-Richtung eine Symmetrie auszeichnet, lässt sich das Problem auf eine Dimension reduzieren. Für die Feldstärke in der Beugungsebene gilt also:
\begin{equation}
E(k_x,k_y) \propto \int_\mathbb{R} \mathrm{d}x \left(\chi_{[9d,10d]}(x) + \chi_{[-10d,-9d]}(x) \right)\exp(-ik_x x)
\end{equation}
Folgende leicht zu zeigende Identität möchten wir gleich benutzen:
\begin{equation}
\int_{-\epsilon}^{+\epsilon}\mathrm{d}\mu\exp(-it\mu) = \frac{2}{t}\sin(\epsilon t)
\label{eq:ident}
\end{equation}
Berechnen wir zunächst das Fourierintegral von oben:
\begin{align}
&\int_\mathbb{R} \mathrm{d}x \left(\chi_{[9d,10d]}(x) + \chi_{[-10d,-9d]}(x) \right)\exp(-ik_x x) \\
&= \int_{-10d}^{-9d}\mathrm{d}x\exp(-ik_x x) +\int_{9d}^{10d}\mathrm{d}x\exp(-ik_x x) \\
&= \int_{-10d}^{10d}\mathrm{d}x\exp(-ik_x x) -\int_{-9d}^{9d}\mathrm{d}x\exp(-ik_x x) \\
&= \frac{2}{k_x} \left[ \sin(10k_x d) - \sin(9k_x d) \right]
\end{align}
Dabei wurde im letzten Schritt \vref{eq:ident} verwendet.
Für das gesuchte Beugungsbild gilt also:
\begin{equation}
I(k_x,k_y) \propto \frac{\left[ \sin(10k_x d) - \sin(9k_x d) \right]^2}{k_x^2}
\end{equation}
Der Graph ist in \vref{fig:fourier} veranschaulicht. 
\begin{figure}[htbp]
\centering
\input{beugung.tex}
\caption{Beugungsbild für beliebiges festes $k_y$ in Einheiten von $d\cdot k_x$.}
\label{fig:fourier}
\end{figure}

\item \vref{fig:fourier} zeigt einen schnell und einen langsam oszillierenden Anteil. Dabei verursacht die Beugung am jeweiligen Einzelspalt den schnell oszillierenden Anteil, während der langsam oszillierende Anteil daher rührt, dass Interferenz zwischen den beiden Spalten stattfindet.

\item Sei $\tilde{I}(k_x,k_y)$ diejenige Intensität, die in der Beugungsebene ankäme, wenn kein Beugungsobjekt vorhanden wäre. Da das Licht parallel einfällt und kohärent ist, gilt $\tilde{I}(k_x,k_y) \equiv \tilde{I}_0$ konstant. Für das inverse Beugungsmuster gilt dann:
\begin{equation}
I^{-1} (k_x,k_y) = \tilde{I}_0 - I (k_x,k_y)
\end{equation}
Das Muster in \vref{fig:fourier} wäre also an der $x$-Achse gespiegelt und in positive $I$-Richtung verschoben.

\end{enumerate}