\section*{\nr.3 \titthree (10 Punkte)}
\begin{enumerate}[(a)]
\item Es gilt $\varphi = 2\pi\frac{\Delta s}{\lambda}$ und $\Delta s = \sin{\alpha }g$. Daraus folgt
\begin{equation}
	\varphi = 2\pi\sin{\alpha}\frac{g}{\lambda}.
\end{equation}
\item Die k-te Welle hat eine Phasenverschiebung von $n\varphi$ (gezählt von 0 bis n-1). Daraus ergibt sich für die k-te Welle die komplexe Darstellung $e^{ik\varphi}$. Damit folgt
\begin{equation}
	E = E_0 \sum_{k=0}^{n-1} e^{ik\varphi} = E_0 \frac{e^{in\varphi} - 1}{e^{i\varphi}  - 1} = E_0 e^{i\frac{n-1}{2}\varphi}\frac{\sin{\frac{n}{2}\varphi}}{sin{\frac{\varphi}{2}}}.
\end{equation}
Da $|e^{i\frac{n-1}{2}\varphi}| = 1$ gilt
\begin{equation}
	I \propto |E|^2 = E_0^2 \frac{\sin^2{\frac{n}{2}\varphi}}{\sin^2{\frac{\varphi}{2}}}
\end{equation} \\
% @ Raphael, hier die Skizze, bau es ein wie es dir passt!
% @ Kianusch, danke, auch wenn du das niemals lesen wirst :P

\begin{figure}[htbp]
\centering
\input{beugung-Gitter.tex}
\caption{Intensitätsverteilung bis zur zweiten Ordnung bei $n$ Spalten, absolut normiert und zur Vergleichbarkeit untereinander durch $n^2$ geteilt.}
\label{fig:beugung}
\end{figure}
\end{enumerate}
