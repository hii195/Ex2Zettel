\section*{\nr.1 \titone (10 Punkte)}
\begin{enumerate}[(a)]
\item Es gilt $I = I_0 sin^2 \Theta$, desweiteren ist außerdem bekannt, dass
\begin{equation}
	P = \int I r^2d\Omega = \frac{8}{3} \pi r^2 I_0.
\end{equation}
Daraus folgt
\begin{equation}
	I = \frac{3 P}{8 \pi r^2} sin^2 \Theta.
\end{equation}
Für $r = \SI{120}{\kilo \metre}$ und $\Theta = \frac{\pi}{2}$ ergibt sich $I = \SI{4,144E-6}{\watt}$.
\item Wegen $I = c\epsilon_0 E^2$ folgt
\begin{equation}
	E = \sqrt{\frac{I}{c \epsilon_0}} = \SI{3,95E-2}{\volt\per\metre}
\end{equation}
\end{enumerate}
