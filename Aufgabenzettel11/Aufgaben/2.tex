\section*{\nr.2 \tittwo (10 Punkte)}
\begin{enumerate}[(a)]
\item Es gilt mithilfe einfacher Geometrie
\begin{equation}
  \sin \alpha = \frac{\delta}{x}
\end{equation}
wobei sich mit $\delta = \frac{\lambda \varphi}{2\pi}$ ergibt, dass
\begin{equation}
  \varphi = \frac{2\pi x \sin \alpha}{\lambda}.
\end{equation}
\item Es ergibt sich
\begin{align}
P(x)&=\langle E_1(t)E_2^{*}(t) \rangle\\
    &=\langle E_0\exp(ik(\bar x+\delta)-i\omega t)E_0\exp(-ik(\bar x)+i\omega t) \rangle\\
    &=\langle E_0^2\exp(ik\sin \alpha x) \rangle\\
    &=E_0^2 \cos (k\sin(\alpha)x)
\end{align}
\begin{figure}[htbp]
\centering
\includegraphics{plot.pdf}
\caption{Verlauf des Signals.}
\label{fig:signal}
\end{figure}

Dieser Verlauf ist in \vref{fig:signal} veranschaulicht.

\item Es gilt
\begin{align}
  P(x)=P_1(x)+P_2(x)&= E_0^2\cos(k\alpha_1x)+E_0^2\cos(k\alpha_2x)\\
    &=2E_0^2\cos \left( \frac{kx}{2}(\alpha_1-\alpha_2)\right)\cos \left( \frac{kx}{2}(\alpha_1+\alpha_2) \right)
\end{align}

\item Die Schwebung lässt sich am besten nachweisen, wenn x die halbe Einhüllende der Schwebung ist, woraus folgt, dass
\begin{equation}
  \Delta \alpha = \frac{c}{\nu x}
\end{equation}

\item  Damit ergibt sich eine Winkelauflösung für das Very Long Baseline Array von $\Delta \alpha = 0.0003$.

\end{enumerate}
