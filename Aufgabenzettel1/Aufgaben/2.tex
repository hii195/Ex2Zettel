\section*{\nr.2 \tittwo (10 Punkte)}
\begin{enumerate}[(a)]
\item
\begin{enumerate}[(1)]
\item Die Kraft, die von einer Ladung auf die Testladung wirkt, wirkt abstoßend und entlang der Verbindungslinie zwischen den beiden
\begin{equation}
  \vec F_i=-a \frac{\vec r_i}{r_i^3}
\end{equation}
mit $a=\frac{2q}{4\pi\epsilon_0}$ und $\vec r_i$ als Ortsvektor der entsprechenden Ladung $i$. Der Einfachheit halber seien im folgenden die Ladungen gegen dem Uhrzeigersinn, beginnend mit der obersten Ladung von $1-5$ numeriert.
Die Felder der beiden Ladungen auf der $y$-Achse gleichen sich im Ursprung gegenseitig aus, es genügt also die verbleibenden drei Ladungen zu betrachten. 

\end{enumerate}
\end{enumerate}