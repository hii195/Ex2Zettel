\section*{\nr.1 \titone (10 Punkte)}
\begin{enumerate}[(a)]
\item Auf den Ringabschnitten befindet sich eine Ladung von
\begin{equation}
	Q=ls
\end{equation}
es ergibt sich also das Verhältnis
\begin{equation}
  \frac{Q_1}{Q_2}= \frac{ls_1}{ls_2} = \frac{s_1}{s_2}.
\end{equation}
Von den Ringabschnitten wird jeweils ein Feld der Stärke
\begin{equation}
  E= \frac{1}{4\pi\epsilon_0}\frac{Q}{r^2}
\end{equation}
erzeugt, es ergibt sich also ein Verhältnis von
\begin{equation}
  \frac{E_1}{E_2}=\frac{Q_1}{Q_2}\frac{r_2^2}{r_1^2}=\frac{s_1}{s_2}\frac{s_2^2}{s_1^2}=\frac{s_2}{s_1}
\end{equation}
und da $s_1<s_2$ erzeugt der Ringabschnitt $s_1$ im Punkt $P$ ein stärkeres elektrisches Feld.

\item Es wäre dann gleich $0$, da die Felder sich ausgleichen würden.
\begin{equation}
  \frac{E_1}{E_2}=\frac{Q_1}{Q_2}\frac{r_2}{r_1}=1
\end{equation}

\item Die Ladung wäre dann 
\begin{equation}
  Q=\rho s^2
\end{equation}
was ein Ladungsverhältnis von 
\begin{equation}
  \frac{Q_1}{Q_2}= \frac{s_1^2}{s_2^2}
\end{equation}
bedeuten würde. Diese würden jeweils ein Feld von
\begin{equation}
  E=\frac{1}{4\pi\epsilon_0}\frac{\rho s^2}{r^2}
\end{equation}
erzeugen.
Im Verhältnis ergibt das
\begin{equation}
  \frac{E_1}{E_2}=1, 
\end{equation}
was bedeutet, dass sich die Felder ausgleichen würden.
Würde sich das Feld jedoch mit $1/r$ ändern, wäre das Gesamtfeld, das durch die beiden Flächenelemente erzeugt wird
\begin{equation}
  E = E_1-E_2 = \frac{\rho}{4\pi\epsilon_0}\left(\frac{s_1^2}{r_1^2}-\frac{s^2}{r_2^2}\right)    
\end{equation}  


\end{enumerate}