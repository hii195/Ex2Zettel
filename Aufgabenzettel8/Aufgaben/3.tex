\section*{\nr.3 \titthree (10 Punkte)}
\begin{enumerate}[(a)]
\item Nach der erweiterten Kirchhoffschen Maschenregel gilt:
\begin{equation}
\hat{U}_e = R \hat{I} +\frac{1}{i\omega C}\hat{I} \iff \hat{I} =\frac{\hat{U}_e}{R+1/(iwC)}
\end{equation}

Setzen wir für die Eingangsspannung also $U_e=\hat{U}_e e^{i\omega t}$ an, ergibt sich der Strom durch $I=\hat{I} e^{i\omega t}$. Die Ausgangsspannung ist die Spannung über dem Kondensator, also:
\begin{equation}
U_a = \frac{1}{i\omega C}I = \frac{\hat{U}_e}{1+i\omega RC} e^{i\omega t}
\end{equation}
Für die Amplitude der Ausgangsspannung ergibt sich
\begin{equation}
|\hat{U}_a| = \hat{U}_e \left| \frac{1-i\omega RC}{1+(\omega R C)^2} \right| = \frac{\hat{U}_e}{\sqrt{1+(\omega R C)^2}},
\end{equation}
während für die Phasenverschiebung bei Betrachtung von Real- und Imaginärteil von $\hat{U}_a$ die Beziehung
\begin{equation}
\tan{\phi} = -\omega R C \iff \phi = -\arctan{\omega R C}
\end{equation}
gilt.
\item
Für das Verhältnis der Spannungsamplituden gilt mit $\omega_0 := 1/(RC)$:
\begin{equation}
\frac{\hat{U}_a}{\hat{U}_e} = \frac{1}{\sqrt{1+(\omega R C )^2}} = \frac{1}{\sqrt{1+(\omega/\omega_0)^2}}
\end{equation}
Für die Phasenverschiebung findet sich analog:
\begin{equation}
\phi = -\arctan (\omega/\omega_0)
\end{equation}
Abbildungen der funktionalen Zusammenhänge sind durch \vref{fig:tiefpassfig} gegeben.
\begin{figure}[htbp]
\centering
\input{output-plotU.tex}\hfill
\input{output-plotphi.tex}
\caption{Links das Verhältnis $\hat{U}_a/\hat{U}_e$, rechts die Phasenverschiebung, jeweils in logarithmischer Skalierung der $x$-Achse.}
\label{fig:tiefpassfig}
\end{figure}

\item Bei kleinen Frequenzen kann sich der Kondensator teilweise aufladen, wodurch sich eine Potentialdifferenz am Ausgang aufbaut. Das Verhältnis aus Ausgangs- und Eingangsspannung ist dadurch hoch. Bei höheren Frequenzen wirkt der Kondensator als bloßes Kabel, wodurch das obige Verhältnis sinkt.

Betrachtet man den Graphen der Phasenverschiebung, so ist sie negativ. Die Ausgangsspannung eilt also der Eingangsspannung voraus, und zwar umso weiter, je höher die Frequenz ist. 

\end{enumerate}