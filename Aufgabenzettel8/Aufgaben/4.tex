\section*{\nr.4 \titfour (10 Punkte)}
\begin{enumerate}[(a)]
\item Für Motor 1 ergeben sich die Scheinleistung $S = I_{eff}U_{eff} = \SI{575}{\watt}$, Wirkleistung $P = cos(\varphi_1) S =\SI{373,75}{\watt}$ und Blindleistung $Q = \sqrt{S^2 - P^2} = \SI{436,96}{\watt}$.
Entsprechend für Motor 2 die Scheinleistung $S = I_{eff}U_{eff} = \SI{805}{\watt}$, Wirkleistung $P = cos(\varphi_2) S =\SI{684,25}{\watt}$ und Blindleistung $Q = \sqrt{S^2 - P^2} = \SI{42406}{\watt}$.
\item $cos(\varphi_{tot}) =\frac{P_1 + P_2}{S_1 + S_2} = 0,62$.
\item I = $\sqrt{2}\SI{6}{\ampere}\exp{iarccos(0,62)} \exp{iwt}$.
Die Scheinleistung ergibt sich als $S =\SI{6}{\ampere}\SI{230}{\volt} = \SI{1380}{\ampere}$, damit ist die Wirkleistung gegeben durch $P = cos(\varphi_{tot})S = \SI{855,6}{\watt}$ damit ist die Blindleistung $Q = \SI{1082,75}{\watt}$.
\end{enumerate}
