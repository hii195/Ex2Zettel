\section*{\nr.1 \titone (10 Punkte)}
\begin{enumerate}[(a)]
\item
Es gilt, da der Strom im recht und linken Teil der RC-Brüke gleich ist, dass Î$_1$ = Î$_2$, der Strom im rechten bzw. linken Teil. Wegen Û$_0$ = Û$_C$ + Û$_R$, die am Kondensator bzw. am Widerstand abfallende Spannung, und dem Ohmschen Gesetz für Impedanzen Û = Î$\hat{Z}$ gilt Û$_C$ = $\hat{Z}_C$ Î$_1$ und Û$_R$ = $\hat{Z}_R$ Î$_1$. Wegen derMaschenregel gilt Û$_0$ = Û$_C$ + Û$_R$ = Î$_1$($\hat{Z}_C$ + $\hat{Z}_R$), also Î$_1$ = $\frac{\hat{U}_0}{\hat{Z}_C + \hat{Z}_R}$. Damit ergibt sich
\begin{equation}
	\Delta \hat{U} = \hat{U}_R - \hat{U}_C = Î_1(\hat{Z}_R - \hat{Z}_C) = \hat{U}_0\frac{\hat{Z}_R - \hat{Z}_C}{\hat{Z}_R + \hat{Z}_C} = \hat{U}_0\frac{1 - \frac{1}{i\omega RC}}{1 + \frac{1}{i\omega RC}}.
\end{equation}
\item Ein Phasenunterschied von $\frac{\pi}{2}$ bedeutet $\Delta \hat{U} = e^{i\frac{\pi}{2}}\hat{U}_0 = i\hat{U}_0$. Also muss gelten
\begin{equation}
	i = \frac{1 - \frac{1}{i\omega RC}}{1 + \frac{1}{i\omega RC}} \Leftrightarrow 1 + \frac{i}{\omega RC} = i + \frac{1}{\omega RC} \Leftrightarrow \omega RC (1 - i) = (1 - i) \Leftrightarrow \omega RC = 1.
\end{equation}
\end{enumerate}
