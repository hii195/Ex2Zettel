\section*{\nr.2 \tittwo (10 Punkte)}
\begin{enumerate}[(a)]
\item Mit
\begin{equation}
  L=\frac{\mu\mu_0N^2A}{l}
\end{equation}
ergeben sich $L_1=1.62\mathrm{H}$ und $L_2=0.005\mathrm{H}$.
\item Es gilt das Ohmsche Gesetz
\begin{equation}
  \hat{U}=\hat{R}\hat{I}
\end{equation}
womit sich 
\begin{equation}
  I_1(t)=\frac{U_1}{\omega L}\sin \omega t
\end{equation}
ergibt.
Mit angegebener Formel ergibt sich für die mittlere Leistung
\begin{equation}
  \bar{P}=f\int_0^{\frac{1}{f}}\frac{U_1^2}{2\omega L}\sin (2\omega t) \mathrm{d}t=0
\end{equation}
\item Mit $\phi =LI$ und den Formeln für den unbelasteten Transformator ergibt sich
\begin{equation}
  M_{12}=\frac{\phi_1}{I_2}=L_1 \frac{I_1}{I_2}=L_1 \frac{N_2}{N_1}=\frac{\mu\mu_0N_1N_2A}{l}=L_2 \frac{N_1}{N_2}=\frac{\phi_2}{I_1}
\end{equation}
außerdem ist
\begin{equation}
  M_{12}=\frac{\mu\mu_0N_1N_2A}{l}=\sqrt{\frac{\mu\mu_0N_1^2A}{l}\frac{\mu\mu_0N_2^2A}{l}}=\sqrt{L_1L_2}
\end{equation}

\item Nach den Gleichungen für den unbelasteten Transformator gilt
\begin{equation}
  U_2(t)=\frac{N_2}{N_1}U_1(t)=
\end{equation}

\item Es gilt
\begin{equation}
  U_2(t)=RI_2(t)
\end{equation}
mit 
\begin{equation}
  U_2(t)=N_2 \dot{\phi}.
\end{equation}
Da aber auch
\begin{equation}
  U_1(t)=N_1\dot{\phi}
\end{equation}
gilt, folgt direkt, dass
\begin{equation}
  I_2(t)=\frac{N_2}{RN_1}U_1(t)=\frac{U_1N_2}{RN_1}\cos \omega t.
\end{equation}
und damit
\begin{equation}
  \dot{I}_2(t)=\frac{\mathrm{d}}{\mathrm{d}t} I_2(t)=-\frac{U_1N_2}{RN_1}\omega \sin \omega t.
\end{equation}

Das Feld in der Ringspule ist gegeben durch
\begin{equation}
  \phi=\frac{L_1}{N_1}I_1-\frac{L_2}{N_2}I_2
\end{equation}
woraus folgt, dass
\begin{equation}
  \dot{\phi}=\frac{L_1}{N_1}\dot{I_1}-\frac{L_2}{N_2}\dot{I_2}=\frac{R}{N_2}I_2(t),
\end{equation}
wobei das negative Vorzeichen aus der Definition der Richtung der Ströme kommt, in diesem Fall in unterschiedliche Richtungen.
Daraus folgt, dass 
\begin{equation}
  \dot{I}_1(t)=\frac{N_1}{L_1}\left(\frac{R}{N_2}I_2(t) + \frac{L_2}{N_2}\dot{I}_2(t)\right)=\frac{U_1}{L_1}\cos \omega t - \frac{L_2U_1}{L_1R}\omega \sin \omega t.
\end{equation}
Damit ergibt sich schließlich
\begin{equation}
  I_1(t)=\int \dot{I}_1(t) \mathrm{d}t=\frac{U_1}{\omega L_1} \sin \omega t+\frac{L_2U_1}{L_1R}\cos \omega t.
\end{equation}
\item Die Leistung ergibt sich als
\begin{align}
  P&=\frac{\omega}{2\pi}\int_{0}^{\frac{2\pi}{\omega}}U_1(t)I_1(t)\mathrm{d}t\\
  &=\frac{\omega}{2\pi}\int_{0}^{\frac{2\pi}{\omega}}U_1 \cos \omega t\left(\frac{U_1}{\omega L_1} \sin \omega t+\frac{L_2U_1}{L_1R}\cos \omega t\right)\mathrm{d}t\\
  &= \frac{L_2U_1^2}{2L_1R}.
\end{align}

\end{enumerate}
