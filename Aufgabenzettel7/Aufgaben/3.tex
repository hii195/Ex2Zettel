\section*{\nr.3 \titthree (10 Punkte)}
\begin{enumerate}[(a)]
\item Aufgrund der Lorentzkraft fließt der Strom in die Richtung $\vec{e}_x \times \vec{e}_z =-\vec{e}_y$, in der Skizze also im Uhrzeigersinn.
\item In der Leiterschleife zeigt das induzierte Magnetfeld in die Zeichenebene hinein, ist also dem von außen angelegten Magnetfeld entgegengesetzt, wie man es von der Lenzschen Regel erwartet.
\item Ist die gesamte Leiterschleife in das homogene (!) Magnetfeld eingedrungen, spürt die Schleife keinen zeitlich veränderlichen magnetischen Fluss, also wird kein Strom induziert.

\item Die vom Magnetfeld durchsetzte Fläche der Leiterschleife beträgt $A(x)=lx$. Solange $x<l$, gilt für den magnetischen Fluss dann:
\begin{equation}
\Phi_m = \int_A \vec{B}\mathrm{d}\vec{A'} = \int_A B\,\mathrm{d}A'=\int_0^{x} Bl\, \mathrm{d}x' = B l x
\end{equation}
Für dessen zeitliche Änderungsrate folgt $\mathrm{d}\Phi_m /\mathrm{d}t = Blv$, was der negativen induzierten Spannung entspricht. Für den induzierten Strom gilt für $x<l$:
\begin{equation}
I (v) = \frac{|U_\text{ind}|}{R} = \frac{Blv}{R}
\end{equation}
\item Einerseits wirkt eine Komponente der Lorentzkraft, welche die gesamte Leiterschleife in $y$-Richtung zu schieben sucht. Diese Komponente interessiert uns hier aber nicht, stattdessen nehmen wir an, dass eine Zwangskraft sicherstellt, dass die Leiterschleife nur in $x$-Richtung beweglich ist. Der induzierte Strom erzeugt eine bremsende Lorentzkraft-Komponente mit
\begin{equation}
\vec{F} = I (\vec{l} \times \vec{B})=-IlB \hat{e}_x =  -\frac{B^2l^2v}{R} \hat{e}_x.
\end{equation}

\item Die Bewegung in $x$-Richtung gehorcht der Bewegungsgleichung:
\begin{equation}
m \dot{v} = -\frac{B^2l^2}{R}v
\end{equation}
Mit der geforderten Anfangsbedingung führt das zur Lösung:
\begin{equation}
v(t) = v_0 \exp{\left(-\frac{B^2l^2}{Rm}t \right)}
\end{equation}
\end{enumerate}