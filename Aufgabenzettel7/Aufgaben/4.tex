\section*{\nr.4 \titfour (10 Punkte)}
\begin{enumerate}[(a)]
\item Es gilt nach dem Induktionsgesetz
\begin{equation}
  \oint \vec E \vec{\mathrm{d}s} = -\frac{\mathrm{d}}{\mathrm{d}t}\int\bar{B}\mathrm{d}A
\end{equation}
aus Symmetriegründen fallen die Vektoren weg und es ergibt sich
\begin{equation}
  2E\pi r=-\dot{\bar{B}}\pi r^2 \implies E(t)=-\dot{\bar{B}}\frac{r}{2}
\end{equation}
damit ergibt sich
\begin{equation}
  a(t)=\frac{er}{2m}\dot{\bar{B}}\implies v(t)=\frac{er}{2m}\bar{B}\implies p=\frac{r}{2}e \bar{B}.
\end{equation}

\item Damit die Elektronen auf einer Kreisbahn bleiben muss gelten
\begin{equation}
  F_Z=F_L \implies m \frac{v^2}{r}=evB_{st}
\end{equation}
mit (a) ergibt sich
\begin{equation}
  \frac{B_{st}}{\bar{B}}=\frac{1}{2}
\end{equation}

\item Mit
\begin{equation}
  \bar{B}(t) = \bar{B}_{max}\sin (\omega t)
\end{equation}
und $\omega=2\pi \cdot 50 \mathrm{Hz}$, ergibt sich
\begin{equation}
  v(t)=\frac{er}{2m}\bar{B}_{max}\sin (2\pi f t)
\end{equation}
woraus folgt, dass
\begin{equation}
  t=\arcsin \left(\frac{2vm}{re\bar{B}_{max}} \right)\frac{1}{2\pi f}=1.09 \cdot 10^{-5} \mathrm{s}.
\end{equation}
\end{enumerate}