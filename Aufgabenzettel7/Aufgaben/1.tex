\section*{\nr.1 \titone (10 Punkte)}
\begin{enumerate}[(a)]
\item Wir legen im Folgenden die beschriebene Anordnung in ein zylindrisches Koordinatensystem, wobei die $z$-Achse mit der Symmetrieachse des Zylinders übereinstimmen soll. Da $l>>r$, sind die $\vec{B}$- und $\vec{H}$-Felder parallel zur Mantelfläche des Zylinders und hängen aus Symmetriegründen nur vom Abstand $a$ zur Symmetrieachse ab, also bei geeigneter Stromrichtung $\vec{B}=B(a)\hat{e}_z$ und $\vec{H}=H(a)\hat{e}_z$. Dann gilt aber $B_\perp=H_\perp=0$. 

Wählen wir nun den in \vref{fig:Intweg} dargestellten Integrationsweg $\mathcal{C}$, wobei wir die gepunkteten Abschnitte hinreichend groß wählen, dass das Magnetfeld vernachlässigbar klein wird:
\begin{equation}
NI =\oint_\mathcal{C} \vec{H}\mathrm{d}\vec{s} \approx \int_0^{l}{H \mathrm{d}s} = Hl
\end{equation}
Also gilt $H=NI/l$, und zwar sowohl im Luftspalt als auch im Eisenkern, da die Parallelkomponente von $\vec{H}$ stetig ist.
\begin{figure}[htbp]
\centering
\begin{tikzpicture}
%Spule mit Beschriftungen
\draw[thick] (0,0) rectangle (3,5);
\node[above] at (1.5,0){Spule};
\node[right] at (3,2.5){$l$};
\draw[->] (1.5,4) --node[above]{a} (2.5,4);
%Integrationsweg
\begin{scope}[red]
\draw (4.5,5) -- (2.5,5) -- (2.5,0) -- (4.5,0);
\draw[dotted,thick] (4.5,0) -- (5.5,0);
\draw (5.5,0) -- (6.5,0) --node[left]{$\mathcal{C}$} (6.5,5) -- (5.5,5);
\draw[dotted,thick] (4.5,5) -- (5.5,5);
\end{scope}
\end{tikzpicture}
\caption{Integrationsweg für das $\vec{H}$-Feld in der Spule schließt im Unendlichen.}
\label{fig:Intweg}
\end{figure}


Für das $\vec{B}$-Feld gilt im Luftspalt $B=\mu_0 H = \mu_0 N I/l$ und im Eisenkern $B=\mu\mu_0 H = \mu\mu_0 N I /l$.

\item Für den magnetischen Fluss durch eine Windung mit Querschnitt $A$ gilt:
\begin{align}
\Phi_m &= \oint_A \vec{B}\mathrm{d}\vec{A'} = \int_A B \,\mathrm{d}A' = \int_\text{Fe} B \,\mathrm{d}A' + \int_\text{Luft} B \,\mathrm{d}A' \\
&= \frac{\mu\mu_0 N I }{l} \pi r_K^2 + \frac{\mu_0 N I}{l} (\pi r^2-\pi r_K^2)\\
&= \frac{\mu_0 N \pi}{l}\left[ (\mu-1) r_K^2 +r^2 \right] I
\end{align}
Nach dem Induktionsgesetz gilt dann:
\begin{equation}
U_\text{ind} = -N \frac{\mathrm{d}\Phi_m}{\mathrm{d}t} =- \frac{\mu_0 N^2 \pi}{l}\left[ (\mu-1) r_K^2 +r^2 \right] \frac{\mathrm{d}I}{\mathrm{d}t}
\end{equation}
Ein Vergleich mit $U_\text{ind}=-L \,\mathrm{d}I/\mathrm{d}t$ liefert:
\begin{equation}
L = \frac{\mu_0 N^2 \pi}{l}\left[ (\mu-1) r_K^2 +r^2 \right] 
\end{equation}

\end{enumerate}