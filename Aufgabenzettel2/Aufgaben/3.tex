\section*{\nr.3 \titthree (10 Punkte)}
\begin{enumerate}[(a)]
\item AUfgrund des Superpositionsprinzip, können die Potentiale jedes Punktes auf dem Ring aufsummiert werden und man erhält das Gesamtpotential des Rings. Alle Punkte $\vec{x} $ auf der z-Achse haben zum Mittelpunk des Rings einen Abstand z. Daraus ergibt sich der Abstand vom Ring als $\sqrt{R^2 + z^2}$. Die aufsummierte Ladung auf dem Ring ist gerade Q. Dadurch ergibt sich das Potential entlang der z-Achse
\begin{equation}
\phi(z) = \frac{1}{4\pi \epsilon_{0}} \frac{Q}{\sqrt{R^2 + z^2}}
\end{equation}
Im Zentrum des Rings (z = 0) ist das Potential extremal (siehe (b) E(0) = 0).
\begin{equation}
\phi(0) = \frac{1}{4\pi \epsilon_{0}} \frac{-10^{-9}C}{\sqrt{(2.5\cdot 10^{-2}m)^2}} \approx -359.67 V
\end{equation}
\item Das elektrische Feld ergibt sich als Gradient des Potentials. In einer Dimension entspricht das der Ableitung $\frac{\mathrm{d}}{\mathrm{d}z}$. Daraus folgt
\begin{equation}
E(z) = \frac{\mathrm{d}\phi (z)}{\mathrm{d}z}  = \frac{Q}{4\pi \epsilon_{0}} \frac{-z}{(R^2 + z^2)^{\frac{3}{2}}}.
\end{equation}
Die Richtung des Feldes muss aus Symmetriegründen parallel zur z-Achse verlaufen. Als Vektorfeld geschrieben ergibt sich das E-Feld also als
\begin{equation}
\vec{E}(z) = \frac{Q}{4\pi \epsilon_{0}} \frac{-z}{(R^2 + z^2)^{\frac{3}{2}}} \cdot \vec{e_{z}}.
\end{equation}
\item Das Potential ist an der Stelle z = 0 extremal. Die potentielle Energie eines Teilchens mit der Ladung q ist gegeben durch $E(z) = \phi(z) \cdot q$. Für das elektron bedeutet das, dass
\begin{equation}
E_{pot}(0) = \phi(0) \cdot (-e) = 359.67 eV
\end{equation}
das Maximum der potentiellen Energie darstellt. Um dieses Potential zu überwinden muss gelten $E_{kin}(0) \ge 0$, also aufgrund von Energieerhaltung $E_{kin}(\infty) \ge E_{pot}(0) := E_{max} \approx 359.67 eV \approx 5.63 \cdot 10^{-17} J$.
\item In Aufgabenteil (c) haben wir aus der Energieerhaltung hergeleiten, dass gelten musss
\begin{equation}
E_{kin}(\infty) = \frac{m_{e}}{2} v^2 \ge E_{pot}(0) \Leftrightarrow v \ge \sqrt{\frac{2E_{max}}{m_{e}}} \approx 1.125 \cdot 10^{7} \frac{m}{s}
\end{equation}
mit $m_{e}$ die Elektronenmasse.
\end{enumerate}
