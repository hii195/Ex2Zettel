\section*{\nr.1 \titone (10 Punkte)}
\begin{enumerate}[(a)]
\item Setzen wir  
\begin{equation}
\phi (r) = \frac{Q}{4\pi\epsilon_0 r}
\end{equation}
mit $r\geq R$ für das Potential der Hohlkugel. Offensichtlich verschwindet diese Größe für $r \to \infty$. Um sicherzugehen, dass es sich hierbei wirklich um das gesuchte Potential handelt, leiten wir probeweise ab:
\begin{equation}
-\vec{\nabla} \phi = -\frac{Q}{4\pi\epsilon_0} \vec{\nabla} \frac{1}{r} = \frac{Q}{4\pi\epsilon_0 r^2} \hat{e}_r = \vec{E}(r)
\end{equation}
Dabei erhalten wir für $r\geq R$ das korrekte elektrische Feld, wie wir es von den Kugelschalen aus Aufgabe 1.4 kennen.

Für das Potential der Kugel gilt dann
\begin{equation}
\phi_R = \phi (R) = \frac{Q}{4\pi\epsilon_0 R} = E_R R,
\end{equation} 
wobei $E_R$ für die elektrische Feldstärke auf der Kugeloberfläche steht.

Für die maximale Spannung folgt dann:
\begin{equation}
U = \phi_R -\phi_\text{Erde} = E_R R
\end{equation}
Setzt man die Zahlenwerte ein, so ergibt sich als maximale Spannung $U=\SI{250}{\kilo\volt}$, bis Vorentladungen auftreten, bzw $U=\SI{250}{\kilo\volt}$, bis es zum Funkendurchbruch kommt.

\item 
Nun werde eine kleine Kugel mit Radius $r$ leitend mit der großen Kugel verbunden. Für das Potential auf der kleinen Kugel gilt
\begin{equation}
\phi_r = E_r r.
\end{equation}
Da dieses Potential mit dem auf der großen Kugel übereinstimmen muss, folgt direkt:
\begin{equation}
E_r = \frac{R}{r}E_R
\end{equation}
Da $R>r$, treten Vorentladungen und Funkendurchbrüche immer zuerst auf der kleinen Kugel auf. Für die maximale Spannung gilt somit
\begin{equation}
U = \phi_r = E_r r.
\end{equation}
Einsetzen der Zahlenwerte liefert $U=\SI{1.25}{\kilo\volt}$ (Vorentladung) sowie $U=\SI{1.75}{\kilo\volt}$ (Funkendurchbruch). Die maximale Spannung ist also deutlich herabgesetzt.

Für das elektrische Feld auf der Oberfläche der Hohlkugel gilt
\begin{equation}
E_R = \frac{r}{R} E_r,
\end{equation}
also $E_R = \SI{0.125}{\kilo\volt\per\centi\meter}$ (Vorentladung) oder $E_R = \SI{0.175}{\kilo\volt\per\centi\meter}$ (Funkendurchbruch). 
\end{enumerate}