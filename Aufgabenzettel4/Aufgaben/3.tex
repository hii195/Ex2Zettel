\section*{\nr.3 \titthree (10 Punkte)}
\begin{enumerate}[(a)]
\item
Aus der Beziehung $U_0 - R_i I = U_1$ folgt:
\begin{equation}
R_i = \frac{U_0-U_1}{I} \approx \SI{17}{\milli\ohm}
\end{equation}
Aus $U_1=R_a I$ folgt:
\begin{equation}
R_a = \frac{U_1}{I} \approx \SI{67}{\milli\ohm}
\end{equation}

Ein Schaltplan der Abbildung findet sich in \vref{fig:circuit}.
\begin{figure}[htbp]
\centering
\begin{tikzpicture}[circuit ee IEC]
%Knoten
\node [contact] (oben) at (5,2) {};
\node [contact] (unten) at (5,-1) {};

\draw (0,-1) to [battery={info={$U_0$}}] (0,2) to [resistor={info={$R_i$}}] (oben) to [resistor={info={$R_a$}}] (unten) to (0,-1);

\draw (unten) to (7,-1) to [voltage source={adjustable',info'={$U_1$}}] (7,2) to (oben);

\end{tikzpicture}
\caption{Schaltplan einer Anordnung von einer Batterie mit Innenwiderstand $R_i$ und zugeschaltetem Anlasser. Ein (ideales) Spannungsmessgerät misst die Spannung $U_1$.}
\label{fig:circuit}
\end{figure}

\item Aus $U_0-R_i I = U_1$ folgt mit $R_i=R_a$, dass $U_0-R_a I = U_1$. Andererseits ist $U_1 = R_a I$. Dadurch ergibt sich:
\begin{equation}
U_0 - U_1 = U_1 \iff U_1 = U_0 /2 \approx \SI{6.25}{\volt}
\end{equation}

\item

Betrachten wir zunächst (a): Für die in der Batterie verbrauchte Leistung gilt
\begin{equation}
P_i = (U_0-U_1)I = \SI{375}{\watt},
\end{equation}
während im Anlasser
\begin{equation}
P_a = U_1 I = \SI{1.5}{\kilo\watt}
\end{equation}
verbraucht werden.

Nun zu (b): Für die verbrauchte Leistung in der Batterie gilt
\begin{equation}
P_i = (U_0 - U_1) I = \frac{(U_0-U_1)^2}{R_i} = \frac{(U_0/2)^2}{R_a} = \frac{U_0^2}{4R_a} \approx \SI{586}{\watt},
\end{equation}
und für den Anlasser
\begin{equation}
P_a = U_1 I = U_1^2/R_a = \frac{(U_0/2)^2}{R_a} = \frac{U_0^2}{4R_a} \approx \SI{586}{\watt}.
\end{equation}

Um die Leistung $P_a$ als Funktion von $R_a$ bei festem $R_i$ zu ermitteln, ist das Gleichungssystem
\begin{align}
U_0 - U_1 &= R_i I \\
U_1 &= R_a I
\end{align} 
mit den Unbekannten $U_1$ und $I$ zu lösen. Elementare Umformungen führen zu $I = U_0/(R_a + R_i)$ und $U_1 = R_a U_0 /(R_a + R_i)$.

Für die im Anlasser verbrauchte Leistung gilt dann:
\begin{equation}
P_a = U_1 I = U_0^2 \frac{R_a}{(R_a+R_i)^2}
\end{equation}
Zur Ermittlung des Maximums muss die Bedingung $\partial P_a / \partial R_a = 0$ ausgewertet werden, was auf 
\begin{equation}
\frac{U_0^2(R_a+R_i)-2U_0^2R_a}{(R_a+R_i)^3}=\frac{\partial P_a}{\partial R_a} = 0 \iff R_a = R_i
\end{equation}
führt. Die verbrauchte Leistung im Anlasser wird also maximal, wenn der Widerstand des Anlassers dem Innenwiderstand der Batterie entspricht.
\end{enumerate}