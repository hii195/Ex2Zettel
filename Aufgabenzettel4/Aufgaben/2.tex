\section*{\nr.2 \tittwo (10 Punkte)}
\begin{enumerate}[(a)]
\item Für einen kleinen Abstand $\mathrm{d}r$ gilt
\begin{equation}
  \mathrm{d}R=\rho \frac{\mathrm{d}r}{A(r)}=\frac{\rho}{2\pi r^2}\mathrm{d}r
\end{equation}
Für die Potentialdifferenz gilt also
\begin{align}
  \Delta\phi&=\left|I\int_{r_1}^{r_2}\frac{\rho}{2\pi r^2}\mathrm{d}r\right|\\
  &=\left|I \frac{\rho}{2\pi}\left(\frac{1}{r_2}-\frac{1}{r_1}\right)\right|
\end{align}
Was für den Mann $\Delta \phi_M=79.2V$ und für die Kuh $\Delta \phi_K=235V$ ergibt.
\item Es gilt
\begin{equation}
  I=\frac{U}{R}
\end{equation}
womit sich für den Mann ein Strom von $I_M=0.02A$ und für die Kuh ein Strom von $I_K=0.059A$ ergibt.
\item Damit der Körperstrom tödlich wäre müsste bei dem kritischen Abstand $r_0$ gelten, dass
\begin{equation}
  \frac{\Delta \phi_M}{R}=I_{krit.}=0.05A.
\end{equation}
Nach einigem Umformen erhält man die quadratische Gleichung 
\begin{equation}
  r_0^2+r_0\Delta r- \frac{I}{I_{crit.}}\frac{\rho}{R2\pi}=0
\end{equation}
mit der Lösung (die negative Lösung ergibt physikalisch keinen Sinn)
\begin{equation}
  r_0=-\frac{\Delta r}{2}+\sqrt{\frac{\Delta r^2}{4}+\frac{I}{I_{crit.}}\frac{\rho}{2\pi R}\Delta r}=62.8m.
\end{equation}
Dabei sind $R$ der Körperwiderstand und $\Delta r$ die Schrittweite.

\end{enumerate}