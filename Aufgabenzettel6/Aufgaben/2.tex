\section*{\nr.2 \tittwo (10 Punkte)}
Für das Feld einer langen Spule gilt für das Magnetefeld (in einer Spule konstant) wegen $\oint \vec{B}\cdot d\vec{s} = B\cdot l = \mu_{0}NI$, dass 
\begin{equation}
	B = \frac{\mu_{0}NI}{2\pi R}.
\end{equation}
Im Innenbereich der Spule umschließt ein Weg keinen Strom, also ist $\oint \vec{B}\cdot d\vec{s} = 0$, es gibt also kein B-Feld in der Ringebene im Innenbereich. Im Außenbereich ist die Summe aller eingeschlossenen Ströme 0, da der Strom der in die Spule hineinfließt an fast der Gleichenstelle wieder hinausfließt. Daraus folgt, dass auch hier I (als Summe aller Ströme I$_k$) gleich 0 ist und somit auch $\oint \vec{B}\cdot d\vec{s} = 0$. Daraus folgt wieder, dass die Komponenten des Magnetfelds in der Ringebene im Außenbereich verschwindet.
Zuletzt berachtet man den Anteil des Stroms, der parallel zum Rings fließt. Es ist leicht zu zeigen, dass mit r der Radius der Windungen gilt 
\begin{equation}
	I_{||} = \frac{I}{\sqrt{1+\left(\frac{2 \pi r N}{2\pi R}\right)^2}}.
\end{equation}
Der Ausdruck geht für große Windungsdichten n = N/2$\pi$R$^2$ (n  $\gg$ 1/2$\pi$r) gegen 0. Dies trifft für die meisten gebräuchlichen Spulen zu. Damit geht auch das B-Feld gegen 0, da es proportional zu I ist. Damit geht das Gesamtfeld unter der Annahme einer hohen Windungsdicht außerhalb des Rings gegen 0 und es gibt nur ein Feld innerhalb der Ringspule.
