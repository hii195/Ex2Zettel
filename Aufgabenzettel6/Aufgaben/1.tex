\section*{\nr.1 \titone (10 Punkte)}
\begin{enumerate}[(a)]
\item Es gilt 
\begin{equation}
  R_H=\frac{U_Hd}{IB}=-5.2\cdot10^{-11}\frac{\mathrm{m}^3}{\mathrm{C}}
\end{equation}
und
\begin{equation}
  n=\frac{1}{R_H}=-1.2\cdot10^{29} \frac{1}{\mathrm{m}^3}
\end{equation}

\item Der Minuspol wird, entsprechend der Skizze auf dem Aufgabenblatt, vorne, der Pluspol hinten sein.

\item In einem Kubikmeter Kupfer befinden sich $1.4\cdot 10^{29}$ Atome, in etwa steuert also jedes Kupferatom ein Leitungselektron bei.

\item Es gilt 
\begin{equation}
\mu=\frac{\sigma}{qn}=-0.003 \frac{\mathrm{m}}{\mathrm{s}}
\end{equation}

\end{enumerate}